\chapter{Discussion}
\label{cha:Discussion}

%development of specific domains

The findings offer valuable insights into the development of motor skills,  particulary piano playing skills, among older individuals. The notion that older people can still develop motor skill could be supported through the context of piano performance. 
Specifically, the study unveiled positive advancements in the domaincs of articulation and dynamics. These improvements are intriguing as they encompass techinical factes often associated with precision and control (TRUE OR NOT?). The increase in expressivity, although subtle, also aligns with these developments, as expressivity is often contingent on effective articulation and dynmaics. However, it is intriguing to note that rhythm did not show a similar level of positive development. This discrepancy needs further investigation and might be explained by the multifaceted nature of rhythm perception and production, which potentially involves cognitive factors that interact differently with aging.
In contrast, the participants' piano playing fluency showed a negative trend over the year of practice. It is worth considering the methodological aspect of this finding. After 12 months, participants were recorded playing both versions of "Ode to Joy". While one version was more difficult than the other, both versions contained the same song, differing only in small nuances. That could have led to confusion between the versions and potentially resulted in poorer performance on the easier version than after three months. Similar logic could apply to the pitch consistency. Further studies should carefully select musical pieces for observation and ensure clearer differentiation more among them. 
The variations in developmental trajectories across the different domains underscore the complexity of the acquistion of musical skill. Each domain seems to require unique cognitive, sensory, and motor demands, that should be further looked into.



Expressivitiy on midi keyboards - musically speaking.....//
noch längere studie
langzeit?
%auswahlprozess? variablen wahl, skala 1 bis 7 wahl, 


%predicted development
%%Musical development was difficult to predict, with some demographic variables influencing only some musical aspects.

The prediction of musical development proved to be challenging, with certain demographic variables influenciny only specific aspect of musical skill. This suggests that musical development is a multifaceted process influenced by many factors, including individual characteristics, history, and perhaps cognitive or physiological changes associated with agig.

%musicality factor
%%factor analysis indicated a one factor solution, which we call musical g-factor.
The factor analysis allowed us to reduce the dimensionality of the data and summarize the information from the six individual musical variables into a single composite measure. This approach provides a more parsimonious representation of musical abilities and enables a deeper understanding of the underlying construct. The analysis yieled a noteworthy result -- a single factor solution, we termed musical g-factor \cite{Pausch2022}. The factor loadings suggest that \textit{musicality} is a multdimensional construct, incorporationg various aspects of musical abilties. The presence of this overarching factor suggests the interconnectivty of musical skills. The identification of this musical g-factor has implications for the understanding of musical skill development. It suggests that while skills may be domain-specific in nature, they are still governed by a common cognitive substrate linked to musical aptitude. This also highlights the potential for transfer of skills across different domains, where improvements in one domain might positively influence another due to the shared musical g-factor.
The multidimensional nature of \textit{musicality} underscores the importance of developing well-rounded musical skills, encompassing articulation, dynamics, rhythm, fluency, pitch, and expressivity. Music educators may benefit from integrating diverse exercises and activities that target these specific musical dimensions to foster overall musical growth in their students.



Rater system \\

%Inter Rater Variability:

Inter-rater reliability, or the extent to which raters disagree in their assessments, is a crucial aspect of any rating system. In our study, the inter-rater reliability exhibited notable patterns across the different variables. Articulation, pitch, and fluency displayed a higher level of consistency among raters, with inter-rater reliability coefficients exceeding 0.9. Expressivity, dynamics, and rhythm demonstrated a slightly lower but still substantial inter-rater consistency, with coefficients hovering around 0.8.
The reasons behind these variations in consistency could be attributed to the inherent characteristics of each variable. Articulation, pitch, and fluency might be easier for raters to consistently grasp due to their more concrete and quantifiable nature. These elements could be objectively identified and evaluated, leading to fewer discrepancies in interpretation among raters. On the other hand, dimensions like expressivity, dynamics, and rhythm may involve more subjectivity, allowing room for differing perceptions and preferences.
Despite these variations, the inter-rater reliability across the panel remained generally high. The raters exhibited consistent evaluations, and the overall consensus on the recordings was evident. This reinforces the notion that even with differences in interpretation, a rater system can yield reliable outcomes.

%Intra Rater Variability:

Intra-rater reliability refers to the agreement of a single rater's assessments across multiple instances. Our analysis revealed that intra-rater reliability was also influenced by the specific evaluation dimensions. Articulation, pitch, rhythm, and fluency displayed higher levels of agreement within raters, whereas dynamics and expressivity exhibited more fluctuation.
The differences in intra-rater reliability can again be indicative of the complexity and subjectivity associated with certain dimensions. The more concrete variables might be easier for raters to consistently assess, while those involving personal interpretation or emotional resonance may lead to varying responses over time.
Notably, the ICCs indicated moderate to good levels of agreement in the ratings, with some reaching excellent levels. This suggests that the raters maintained a certain level of agreement in their assessments over multiple rounds of evaluation. The variation in ICCs did not depend on educational background or teaching experience. 
Future research could delve deeper into understanding the factors influencing individual raters' tendencies towards agreement or variability. Investigating the correlation between personal traits, background, and the specific dimensions that showed varying levels of reliability could provide valuable insights into the rater's decision-making process.

In conclusion, the rater system demonstrated a satisfactory level of inter-rater and intra-rater reliability. The findings highlight the importance of considering intra- and inter-rater reliability in the evaluation of piano recordings. The consistent assessments of fluency, pitch, rhythm, and articulation suggest that these musical parameters can be considered to be evaluated robust and reliable. However, the variability observed in the evaluations of expressivity and dynamics indicates potential differences in the interpretation and perception of these aspects among the raters and needs further investigation to explore the factors contributing to the variability in evaluations of expressivity and dynamics. The variations observed shed light on the complex interplay between the nature of the evaluated dimensions, and rater subjectivity. 
These findings underscore the significance of well-defined evaluation criteria, continuous communication, and suggest an accurate investigation of the raters' personality and background for achieving reliable and comprehensive assessments.



%raus? Careful decisions were made to evaluate the piano performances, following the model described in chapter \ref*{cap:musicalabilities}, in order to minimize possible sources of friction. The performance context and assessment's purpose were clear and unchanged throughout ratings, allowing for the exclusion of these factors from the analysis. 


