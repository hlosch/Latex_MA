Background: Although the literature has demonstrated that older people are generally still capable of learning new motor skills \cite{Boyke2008, Anguera2013, Altenmuller2020}, the development of musical skills in older people is poorly studied. The present experiment aimed to examine the development of musical abilities in the aging population and to identify factors that contribute to the development of these skills.

Methods: A cohort of 72 elderly participants (age = 69.5 years ± 3.14) from Hannover and Geneva without prior musical practice participated in a 12-month piano training program with weekly lessons. Participants’ piano performance was audibly assessed at 3 and 12 months of intervention and subsequently rated by unrelated piano students (holding at least a bachelor’s degree) with teaching experience. Each rater evaluated the piano performance of each participant blindly and in randomized order, considering articulation, dynamics, rhythm, pitch, fluency and expressivity on a scale from 1 to 7. Various predictors such as age, gender, education, and musical sophistication were considered as potential predictors of individual progress.

Results: Bayesian statistical analysis of the data show that after three months of practice, younger participants with higher CogTel scores and more cognitive reserve achieve higher results in rhythm, fluency and pitch. In articulation (0.29, CI 95\%: [0.08, 0.51]) and dynamics (0.12, CI 95\%: [-0.01, 0.42]) a clear time effect over the course of the intervention can be observed. Especially younger participants with less cognitive reserve and those who claim to have less musical sophistication (such as musical emotions, singing abilities and engagement in musical activities) show greater improvement in their skills over time, catching up to or even surpassing participants who initially performed better.

Discussion: The findings of this study have the potential to advance our understanding of the development of musical skills in elderly individuals and contribute to the development of tailored approaches to promote musical engagement in this population. By identifying the factors that contribute to improved musical training outcomes, targeted interventions can be designed to optimize musical training strategies for older adults.