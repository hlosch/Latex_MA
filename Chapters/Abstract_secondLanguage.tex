The federal government's targets state that Germany should become greenhouse gas neutral by 2045. In 2021, 15\,\% of greenhouse gas emissions in Germany were attributable to the building sector. Innovations in the Building Energy Act and the federal government's requirement to install $500.000$ of new heat pumps every year from 2024 increase the demand for heat pumps.

In 2022, $86,9\,\%$ of the $236.000$ installed heat pumps in Germany were air-water heat pumps, as the investment costs are low compared to heat pumps with other energy sources. Air-to-water heat pumps may experience frosting on the heat exchanger in certain weather conditions, which reduces the Coefficient of Performance of the heat pump. To restore the efficiency of the heat pump, an energy-intensive defrosting process is necessary. The average Coefficient of Performance of the heat pump depends on the defrosting time. By determining optimal defrosting times, the efficiency of the heat pump can be increased. In most commercial heat pumps, the defrosting process is initiated in a time-controlled manner. Demand-based defrost controls are proposed in the literature, which often require additional sensor technology. One example is a control that measures the pressure loss of the air flow at the evaporator and initiates the defrosting process when a threshold value is exceeded.

In this work, a Reinforcement Learning agent for determining optimal defrosting times is developed and tested on an air-to-water heat pump. Since large amounts of data are necessary for the training, training data are generated in a heuristic simulation. An RL agent is developed that is able to abstract the optimisation problem from the simulation and project it onto real test bench data. This is followed by an experimental validation on a Hardware-in-the-Loop test bench at the E.ON Energy Research Center at RWTH Aachen University. Three dynamic 24-h tests with different frost growth rates are carried out. The validation is carried out by comparing a time-based controller (R1), a pressure difference-based controller (R2) designed according to literature recommendations and an optimised pressure difference-based controller (R3). Compared to R1, the average Coefficient of Performance of the RL controller is $7,1\,\%$ higher for fast frost growth, but $0,7\,\%$ lower for slow frost growth. Compared to R2, the average Coefficient of Performance of the RL controller is better between $3,3\,\%$ and $9,1\,\%$. Only R3 is more efficient than the RL control at slow frost growth up to~$2\,\%$.
