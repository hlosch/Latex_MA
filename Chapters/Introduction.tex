\chapter{Introduction}

The aging process is an inevitable and complex journey experienced by every individual. As the world's population continues to age, understanding how to promote healthy aging and preserve cognitive functions becomes a crucial area of scientific inquiry. Engaging in intellectual stimulating activities has shown promise in promoting cognitive vitality and potentially mitigating age-related cognitive decline. Learning to play a musical instrument has been recognized as a beneficial and enjoyable activity with potential cogntive and emotional benefits. Among various musical instruments, the piano stands out for its vertility, encompassing a wide range of tonalities, expressive capabilities, and cognitive demands. Learning to play the piano holds promise for enhancing cognitive function and emotional well-being in older aduts. However, limited research has explored the scientific development of musical abilities in musically untrained elderly individuals over an extended period.\\
While previous research has explored the cognitive benefits of musical training across different age groups, relatively little attention has been pain to the specific effect of piano training in older adults, especially on those musically untrained. This thesis presents the findings of a one-year longitudinal study that aimed to explore the evolution of piano development and musical abilities in musically untrained elderly adults. In order to do so, the study implemented a rater system to objectively measure musical performances, assessing key elements such as articulation, dynamics, rhythm, pitch, fluency and expressivity. 
This study seeks to address different research questions: 
\begin{enumerate}
\item How do musically untrained elderly adults' abilities to play the piano evolve over one year?
\item What are the specific improvements ovserved in different aspects of musical performance?
\item Is there a generalizable \textit{musicality} factor and how does it evolve during the one-year duration?
\item How can a rater system effectively evaluate and measure the progress and proficiency of piano playing?
\end{enumerate}

To explore these questions, a sample of musically untrained elderly adults was recruited and provided with weekly piano lessons. Cognitive assessments, psychological questionnaires, and musical proficiency evaluations were conducted at regular intervals throughout the one-year duration. The data collected was analyzed to examine the changes and progress in musical proficiency over time. Throughout the study, musical performances were evaluated by trained raters using a comprehensive scoring system, encompassing the dimensions of articulation, dynamics, rhythm, pitch, fluency and expressivity. Raters assignes rating on a scale ranging from 1 (poor) to 7 (excellent) for each dimension, providing objective measurements of the participants' musical progress.\\
The significance of this research lies in its ability to shed light of the development of musical abilities in musically untrained elderly individuals engaged in piano training over an extended period. By implementing a rater system for performance evaluation, this study ensures a comprehensive and standardized assessment of musical proficiency, enabling a detailed analysis of the specific aspects of musical performance that show improvement over time.\\
The findings of this research have both academic and practical implications. Academically, this study contributes to the existing literature by proving empirical evidence on the development of musical abilities in musically untrained elderly individuals through piano training. Practically, the use of a rater system for performance evaluation can inform the design and implementation of music-based interventions aimed at enhancing musical proficiency and enjoyment for older adults.\\
In summary, this thesis presents a one-year longitudinal study investigating the evolution of musical abilities in musically untrained elderly individuals engaged in piano training. By employing a rater system to evaluate performances based on articulation, dynamics, rhythm, pitch, fluency, and expressivity, this research aims to provide objective measurements of musical progress and contribute to our understanding of the benefits of piano training in promoting musical development and well-being among older individuals. \\

